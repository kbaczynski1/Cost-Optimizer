\documentclass[11pt]{article}

% Formatting
\usepackage[utf8]{inputenc}
\usepackage{polski}
\usepackage[margin=1in]{geometry}

% Title content
\title{Algorytmy i struktury danych
\\Projekt indywidualny
\bigbreak Specyfikacja implementacyjna}
\author{Kacper Baczyński, 276409}
\date{Warszawa, 2020}

\begin{document}

\maketitle

% Sformułowanie problemu
\section{Sformułowanie problemu}
Przy założeniu warunku równowagi podaży i popytu (wszystkie wyprodukowane szczepionki zostaną dostarczone do aptek):
\begin{equation}
    \sum\limits_{i=1}^{m} a_{i} = \sum\limits_{j=1}^n b_{j},
\end{equation}
zadanie przyjmuje postać - znaleźć minimum:
\begin{equation}
    \sum\limits_{i=1}^{m}\sum\limits_{j=1}^{n}c_{ij}x_{ij},
\end{equation}
przy warunkach:
\begin{equation}
    \sum\limits_{j=1}^{n} x_{ij} = a_{i} \quad (i = 1, 2,..., m),
\end{equation}
\begin{equation}
    \sum\limits_{i=1}^{m} x_{ij} = b_{j} \quad (j = 1, 2,..., n),
\end{equation}
\begin{equation}
    0 \leq x_{ij} \leq o_{ij} \quad (i = 1, 2,..., m; \ j = 1, 2,..., n),
\end{equation}
gdzie:
\begin{itemize}
    \item $ a_{i} $ - dzienna produkcja i-tego producenta,
    \item $ b_{j} $ - dzienne zapotrzebowanie j-tej apteki,
    \item $ c_{ij} $ - cena pojedynczej szczepionki od i-tego producenta dla j-tej apteki [zł],
    \item $ x_{ij} $ - dzienna liczba dostarczonych szczepionek od i-tego producenta do j-tej apteki,
    \item $ o_{ij} $ - dzienna maksymalna liczba dostarczonych szczepionek od i-tego producenta do j-tej apteki,
    \item $ a_{i}, b_{i} $ są nieujemnymi liczbami całkowitymi,
    \item $ c_{ij} $ jest nieujemną liczbą zmiennoprzecinkową.
\end{itemize}
Funkcjonał (2) to funkcja kosztu. Warunek (3) jest ograniczeniem dla dostawców (producentów), a warunek (4) ograniczeniem dla odbiorców (aptek). Warunek (4) jest z kolei warunkiem brzegowym.

Przedstawiany problem można potraktować jako tzw. zagadnienie transportowe (ZT). Jeśli zachodzi równość (1), to zagadnienie transportowe nazywa się zamkniętym.
Jeśli warunek (1) jest niespełniony, tzn.:
\begin{equation}
    \sum\limits_{i=1}^{m} a_{i} < \sum\limits_{j=1}^n b_{j},
\end{equation}
lub
\begin{equation}
    \sum\limits_{i=1}^{m} a_{i} > \sum\limits_{j=1}^n b_{j},
\end{equation}
to zagadnienie transportowe jest otwarte. Aby rozwiązać taki problem należy najpierw sprowadzić otwarte ZT do zamkniętego.

Relacja (6) mówi o tym, że niecałe zapotrzebowanie aptek zostanie zaspokojone przez producentów. Przypadek ten należy sprowadzić do zamkniętego ZT wprowadzając zmienne dodatkowe $x_{m+1,j}, j = 1,..., n$ interpretowane jako liczby szczepionek dostarczone przez fikcyjnego producenta, faktycznie zaś nie dostarczone przez nikogo. Przyjmuje się, że dla każdego $j=1,..., n$ zachodzi równość:
\begin{equation}
    x_{m+1,j}=b_{j}-\sum\limits_{i=1}^{m} x_{ij}.
\end{equation}
W rezultacie otrzymuje się zagadnienie postaci:
\begin{equation}
    \sum\limits_{i=1}^{m+1}\sum\limits_{j=1}^{n}c_{ij}x_{ij},
\end{equation}
przy warunkach:
\begin{equation}
    \sum\limits_{j=1}^{n} x_{ij} = a_{i} \quad (i = 1, 2,..., m+1),
\end{equation}
\begin{equation}
    \sum\limits_{i=1}^{m+1} x_{ij} = b_{j} \quad (j = 1, 2,..., n),
\end{equation}
\begin{equation}
    0 \leq x_{ij} \leq o_{ij} \quad (i = 1, 2,..., m+1; \ j = 1, 2,..., n),
\end{equation}

Z kolei nierówność (7) oznacza, że niecały towar od producentów zostanie dostarczony do aptek. Przypadek ten należy sprowadzić do zamkniętego ZT wprowadzając zmienne dodatkowe $x_{i,n+1}, i = 1,..., m$ interpretowane jako liczby szczepionek odebrane przez fikcyjną aptekę, faktycznie zaś nie odebrane przez nikogo. Przyjmuje się, że dla każdego $i=1,..., m$ zachodzi równość:
\begin{equation}
    x_{i,n+1}=a_{i}-\sum\limits_{j=1}^{n} x_{ij}.
\end{equation}
W rezultacie otrzymuje się zagadnienie postaci:
\begin{equation}
    \sum\limits_{i=1}^{m}\sum\limits_{j=1}^{n+1}c_{ij}x_{ij},
\end{equation}
przy warunkach:
\begin{equation}
    \sum\limits_{j=1}^{n+1} x_{ij} = a_{i} \quad (i = 1, 2,..., m),
\end{equation}
\begin{equation}
    \sum\limits_{i=1}^{m} x_{ij} = b_{j} \quad (j = 1, 2,..., n+1),
\end{equation}
\begin{equation}
    0 \leq x_{ij} \leq o_{ij} \quad (i = 1, 2,..., m; \ j = 1, 2,..., n+1),
\end{equation}

% Algorytm
\section{Algorytm}
\begin{enumerate}
    \item 
\end{enumerate}


\end{document}
