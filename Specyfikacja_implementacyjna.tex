\documentclass[11pt]{article}

% Formatting
\usepackage[utf8]{inputenc}
\usepackage{polski}
\usepackage[margin=1in]{geometry}

% Title content
\title{Algorytmy i struktury danych
\\Projekt indywidualny
\bigbreak Specyfikacja implementacyjna}
\author{Kacper Baczyński, 276409}
\date{Warszawa, 2020}

\begin{document}

\maketitle

% Matematyczne sformułowanie problemu
\section{Matematyczne sformułowanie problemu}
Znaleźć minimum:
$$ y = \sum\limits_{j=1}^{m}\sum\limits_{i=1}^{n}c_{ij}x_{ij}, $$
przy warunkach:
$$ \sum\limits_{i=1}^{n} x_{ij} = z_{j} \quad (j = 1, 2,..., m), $$
$$ \sum\limits_{j=1}^{m} x_{ij} \leq p_{i} \quad (i = 1, 2,..., n), $$
$$ 0 \leq x_{ij} \leq m_{ij} \quad (i = 1, 2,..., n; \ j = 1, 2,..., m), $$
gdzie:
\begin{itemize}
    \item $ c_{ij} $ - cena pojedynczej szczepionki od i-tego producenta dla j-tej apteki [zł],
    \item $ x_{ij} $ - dzienna liczba dostarczonych szczepionek od i-tego producenta do j-tej apteki,
    \item $ z_{j} $ - dzienne zapotrzebowanie j-tej apteki,
    \item $ p_{i} $ - dzienna produkcja i-tego producenta,
    \item $ m_{ij} $ - dzienna maksymalna liczba dostarczonych szczepionek od i-tego producenta do j-tej apteki,
    \item $ p_{i}, z_{i} $ są liczbami naturalnymi (włączając 0),
    \item $ c_{ij} $ jest nieujemną liczbą zmiennoprzecinkową.
\end{itemize}

\end{document}
