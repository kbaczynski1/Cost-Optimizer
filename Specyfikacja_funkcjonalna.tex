\documentclass{article}

% Formatting
\usepackage[utf8]{inputenc}
\usepackage[margin=1in]{geometry}
\usepackage[titletoc,title]{appendix}

% Title content
\title{Algorytmy i struktury danych
\\Projekt indywidualny\\
\bigbreak
Specyfikacja funkcjonalna}
\author{Kacper Baczyński}
\date{Warszawa, 2020}

\begin{document}

\maketitle

% Wymagania dla pliku wejściowego
\section{Wymagania dla pliku wejściowego}
Przykładowy plik wczytywany przez program powinien wyglądać następująco:\\\\
# Producenci szczepionek (id \textbar\vspace{} nazwa \textbar\vspace{} dzienna produkcja)\\
0 \textbar\vspace{} BioTech 2.0 \textbar\vspace{} 900\\
1 \textbar\vspace{} Eko Polska 2020 \textbar\vspace{} 1300\\
2 \textbar\vspace{} Post-Covid Sp. z o.o. \textbar\vspace{} 1100\\
# Apteki (id \textbar\vspace{} nazwa \textbar\vspace{} dzienne zapotrzebowanie)\\
0 \textbar\vspace{} CentMedEko Centrala \textbar\vspace{} 450\\
1 \textbar\vspace{} CentMedEko 24h \textbar\vspace{} 690\\
2 \textbar\vspace{} CentMedEko Nowogrodzka \textbar\vspace{} 1200\\
# Połączenia producentów i aptek (id producenta \textbar\vspace{} id apteki \textbar\vspace{} dzienna maksymalna liczba dostarczanych szczepionek \textbar\vspace{} koszt szczepionki [zł] )\\
0 \textbar\vspace{} 0 \textbar\vspace{} 800 \textbar\vspace{} 70.5\\
0 \textbar\vspace{} 1 \textbar\vspace{} 600 \textbar\vspace{} 70\\
0 \textbar\vspace{} 2 \textbar\vspace{} 750 \textbar\vspace{} 90.99\\
1 \textbar\vspace{} 0 \textbar\vspace{} 900 \textbar\vspace{} 100\\
1 \textbar\vspace{} 1 \textbar\vspace{} 600 \textbar\vspace{} 80\\
1 \textbar\vspace{} 2 \textbar\vspace{} 450 \textbar\vspace{} 70\\
2 \textbar\vspace{} 0 \textbar\vspace{} 900 \textbar\vspace{} 80\\
2 \textbar\vspace{} 1 \textbar\vspace{} 900 \textbar\vspace{} 90\\
2 \textbar\vspace{} 2 \textbar\vspace{} 300 \textbar\vspace{} 100\\

Plik wczytywany przez program powinien:\\
1.	Posiadać trzy sekcje w następującej kolejności: producenci szczepionek, apteki, połączenia producentów i aptek.\\
2.	Każda sekcja powinna zawierać tytuł (może być pusty). Tytuł każdej sekcji musi rozpoczynać się od znaku ‘#’. Po tym znaku sekcja może posiadać swoją nazwę oraz uwzględniać w nawiasie podane dane w tej sekcji (ale nie musi – tytuł pusty).\\
3.	Sekcja pierwsza - producenci szczepionek w kolejnych wierszach musi zawierać dane dotyczące kolejnych producentów. Dane każdego producenta muszą być podane w następującej kolejności: id | nazwa | dzienna produkcja. Środkowa dana, czyli nazwa producenta musi być oddzielona od pozostałych danych znakiem strumienia „|”. Dana id oznacza numer identyfikacyjny producenta i musi być liczbą całkowitą. Każdy producent musi mieć unikalne id (dwóch producentów nie może mieć takiego samego id). Nazwa producenta może zawierać wszystkie znaki oprócz znaku strumienia „|”. Dzienna produkcja na jaką osiąga dany producent musi być liczbą całkowitą.\\
4.	Sekcja druga - apteki w kolejnych wierszach musi zawierać dane dotyczące kolejnych aptek. Dane każdej apteki muszą być podane w następującej kolejności: id | nazwa | dzienne zapotrzebowanie. Środkowa dana, czyli nazwa apteki musi być oddzielona od pozostałych danych znakiem strumienia „|”. Dana id oznacza numer identyfikacyjny apteki i musi być liczbą całkowitą. Ponadto każdy producent musi mieć unikalne id (dwóch producentów nie może mieć takiego samego id). Nazwa apteki może zawierać wszystkie znaki oprócz znaku strumienia „|”. Dzienne zapotrzebowanie danej apteki musi być liczbą całkowitą.\\
5.	Sekcja trzecia - połączenia producentów i aptek w kolejnych wierszach musi zawierać dane dotyczące umowy między danym producentem, a daną apteką. Sekcja ta musi zawierać połączenia (umowy) każdego producenta z każdą apteką. Każdego takie połączenie (wiersz) musi być opisane przez dane w następującej kolejności: id producenta | id apteki | dzienna maksymalna liczba dostarczanych szczepionek | koszt szczepionki [zł]. W każdym wierszu po każdej danej musi wystąpić znak strumienia „|” za wyjątkiem ostatniej danej w wierszu – koszt szczepionki [zł].  Dana „id producenta” oraz „id apteki” oznaczają odpowiednio numer identyfikacyjny producenta i apteki. Obie te dane muszą być liczbą całkowitą. „|”. Dzienna maksymalna liczba dostarczanych szczepionek wynikająca z umowy musi być liczbą całkowitą. Koszt szczepionki po jakiej dany producent sprzedanej szczepionkę danej aptece może być liczbą zmiennoprzecinkową. Koszt ten musi być podany w złotych [zł].\\


\end{document}
