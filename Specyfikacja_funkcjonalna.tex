\documentclass[11pt]{article}

% Formatting
\usepackage{polski}
\usepackage[utf8]{inputenc}
\usepackage[margin=1in]{geometry}
\usepackage{parskip}

% Title content
\title{Algorytmy i struktury danych
\\Projekt indywidualny
\bigbreak Specyfikacja funkcjonalna}
\author{Kacper Baczyński, 276409}
\date{Warszawa, 2020}

\begin{document}

\maketitle

% Wstęp
\section{Wprowadzenie}
Niniejszy dokumnet stanowi specyfikację funkcjonalną programu \emph{CostOptimizer}. Zadaniem programu jest znalezienie dla każdej apteki takich producentów oraz ilości dostarczonych przez nich szczepionek, aby zaspokoić dzienne zapotrzboeanie apteki i zminimalziować koszt całkowity. Przez koszt całkowity rozumie się sumę kosztów tranzakcji wszystkich aptek. 
% Wymagania dla pliku wejściowego
\section{Wymagania dla pliku wejściowego}
Przykładowy plik wczytywany przez program powinien wyglądać następująco:\\*\\*
\framebox{%
  \begin{minipage}{0.98\linewidth}
    \# Producenci szczepionek (id \textbar\space nazwa \textbar\space dzienna produkcja)\\
    0 \textbar\space BioTech 2.0 \textbar\space 900\\
    1 \textbar\space Eko Polska 2020 \textbar\space 1300\\
    2 \textbar\space Post-Covid Sp. z o.o. \textbar\space 1100\\
    \# Apteki (id \textbar\space{} nazwa \textbar\space{} dzienne zapotrzebowanie)\\
    0 \textbar\space CentMedEko Centrala \textbar\space 450\\
    1 \textbar\space CentMedEko 24h \textbar\space 690\\
    2 \textbar\space CentMedEko Nowogrodzka \textbar\space 1200\\
    \# Połączenia producentów i aptek (id producenta \textbar\space id apteki \textbar\space dzienna maksymalna liczba dostarczanych szczepionek \textbar\space koszt szczepionki [zł] )\\
    0 \textbar\space 0 \textbar\space 800 \textbar\space 70.5\\
    0 \textbar\space 1 \textbar\space 600 \textbar\space 70\\
    0 \textbar\space 2 \textbar\space 750 \textbar\space 90.99\\
    1 \textbar\space 0 \textbar\space 900 \textbar\space 100\\
    1 \textbar\space 1 \textbar\space 600 \textbar\space 80\\
    1 \textbar\space 2 \textbar\space 450 \textbar\space 70\\
    2 \textbar\space 0 \textbar\space 900 \textbar\space 80\\
    2 \textbar\space 1 \textbar\space 900 \textbar\space 90\\
    2 \textbar\space 2 \textbar\space 300 \textbar\space 100
  \end{minipage}}

\vspace{2em}
Wymagania dla pliku wczytywnaego przez program:
\begin{enumerate}
    \item Plik wejściowy musi nazywać się "data" i mieć rozszerzenie (*.txt).
    \item Plik wejściowy musi posiadać trzy sekcje w następującej kolejności: \emph{producenci szczepionek, apteki, połączenia producentów i aptek}.
    \item Każda sekcja musi rozpoczynać się od znaku '\#'. Po tym znaku w tym samym wierszu sekcja może (ale nie musi) posiadać swoją nazwę oraz nazwy danych, które zawiera każdy następny wiersz tej sekcji. Po tytule sekcji ("\#...") konieczne jest przejście do nowej lini (znak nowej linii).
    \item Sekcja pierwsza - producenci szczepionek po wierszu tytułowym ("\#...") w kolejnych wierszach musi zawierać dane dotyczące kolejnych producentów. Dane producentów muszą być odzielone znakiem nowej lini (każdy producent to nowy wiersz). Dane każdego producenta muszą być podane w następującej kolejności: \emph{id \textbar\space nazwa \textbar\space dzienna produkcja}. W każdym wierszu po każdej danej musi wystąpić znak strumienia '\textbar' za wyjątkiem ostatniej danej w wierszu - \emph{dzienna produkcja}. Dana \emph{id} oznacza numer identyfikacyjny producenta i musi być liczbą całkowitą. Każdy producent musi mieć unikalne \emph{id} (dwóch producentów nie może mieć takiego samego \emph{id}). \emph{Nazwa} producenta może zawierać wszystkie znaki oprócz znaku strumienia '\textbar'. \emph{Dzienna produkcja} jaką osiąga dany producent musi być liczbą całkowitą.
    \item Sekcja druga - \emph{apteki} po wierszu tytułowym ("\#...") w kolejnych wierszach musi zawierać dane dotyczące kolejnych aptek. Dane aptek muszą być odzielone znakiem nowej lini (każda apteka to nowy wiersz). Dane każdej apteki muszą być podane w następującej kolejności: \emph{id \textbar\space nazwa \textbar\space dzienne zapotrzebowanie}. W każdym wierszu po każdej danej musi wystąpić znak strumienia '\textbar' za wyjątkiem ostatniej danej w wierszu - \emph{dzienne zapotrzebowanie}. Dana \emph{id} oznacza numer identyfikacyjny apteki i musi być liczbą całkowitą. Każda apteka musi mieć unikalne \emph{id} (dwie apteki nie mogą mieć takiego samego \emph{id}). \emph{Nazwa} apteki może zawierać wszystkie znaki oprócz znaku strumienia '\textbar'. \emph{Dzienne zapotrzebowanie} danej apteki musi być liczbą całkowitą.
    \item Sekcja trzecia - \emph{połączenia producentów i aptek} po wierszu tytułowym ("\#...") w kolejnych wierszach musi zawierać dane dotyczące kolejnych umów między danym producentem a daną apteką. Dane połączeń muszą być odzielone znakiem nowej lini (każde połączenie to nowy wiersz). Sekcja ta musi zawierać połączenia (umowy) każdego producenta z każdą apteką. Każde takie połączenie musi być unikalne (nie może powtórzyć się połączenie tego samego producenta z tą samą apteką). Dane każdej umowy muszą być podane w następującej kolejności: \emph{id producenta \textbar\space id apteki \textbar\space dzienna maksymalna liczba dostarczanych szczepionek \textbar\space koszt szczepionki [zł]}. W każdym wierszu po każdej danej musi wystąpić znak strumienia ‘\textbar’ za wyjątkiem ostatniej danej w wierszu - \emph{koszt szczepionki [zł]}.  Dane \emph{id producenta} oraz \emph{id apteki} to odpowiednio numer identyfikacyjny producenta i apteki z wcześniej omawianych sekcji, a więc dane te muszą spełniać wymagania wyszczególnione w tych sekcjach oraz mogą przyjmować tylko takie wartości jakie zostały w tych sekcjach wymienione (nie może być umowy między niezdefiniowaną wcześniej apteką czy prducentem). \emph{Dzienna maksymalna liczba dostarczanych szczepionek} wynikająca z umowy musi być liczbą całkowitą. \emph{Koszt szczepionki [zł]} po jakiej dany producent sprzedaje szczepionkę danej aptece może być liczbą zmiennoprzecinkową. Koszt ten musi być podany w złotych [zł].
\end{enumerate}
% Instrukcja korzystania z programu
\section{Instrukcja korzystania z programu}
W celu użycia programu należy postępować według poniżych kroków:
\begin{enumerate}
    \item Proszę upewnić się, czy na komputerze zainstalowane jest JRE (ang. Java Runtime Environment) - środowisko uruchomieniowe Javy. W tym celu należy otworzyć wiersz poleceń (terminal) i wywołać następującą komendę:\\*\\*
    \framebox{%
        \begin{minipage}{0.98\linewidth}
        java -version
        \end{minipage}}
    
    \vspace{1em}
    Jeśli pojawi się informacja o wersji JRE, to oznacza, że środowisko jest zainstalowane. Jeśli taka informacja się nie pojawi, wówczas konienczna jest instalacja JRE.
    \item Proszę przenieść programu \emph{CostOptimizer.jar} oraz plik wejściowy \emph{data.txt} do wybranej przez siebie lokalizacji (folderu). 
    \item Proszę otworzyć terminal w miejscu, gdzie znajduje się program i plik wejściowy. Można też otworzyć terminal w dowolnym miejscu i zmienić bieżącą lokalizację na lokalizację programu i pliku wejściowego za pomocą polecenia "cd \textless sciezka\_do\_pliku\textgreater", a więc np.:\\*\\*
    \framebox{%
        \begin{minipage}{0.98\linewidth}
        cd C:\textbackslash Katalog programu
        \end{minipage}}
    
    \vspace{1em}
    \item Proszę uruchomić program za pomocą poniżeszj komendy w terminalu:\\*\\*
    \framebox{%
        \begin{minipage}{0.98\linewidth}
        java -jar CostOptimizer.jar data.txt
        \end{minipage}}
    
    \vspace{1em}
    \item Jeśli plik wejściowy spełnia wszystkie wymagania opsiane w rozdziale 2, to programu zwróci plik \emph{optimized.txt}. Plik ten zostanie zapisany w miejscu, gdzie znajduje się program i plik wejściowy. Zawartość pliku wyjściowego jest pokazana i omówiona w rodziale 4.
\end{enumerate}
% Wynik działania programu (plik wyjściowy)
\section{Wynik działania programu (plik wyjściowy)}
Przykładowy plik wyjściowy powinien wygląd następująco:\\*\\*
\framebox{%
  \begin{minipage}{0.98\linewidth}
    BioTech 2.0\hspace{2em} -\textgreater\space CentMedEko Centrala [Koszt = 300 * 70.5 = 21150 zł]\\
    Eko Polska 2020 -\textgreater\space CentMedEko Centrala [Koszt = 150 * 100 = 15000 zł]\\
    Opłaty całkowite: 36150 zł
  \end{minipage}}

\vspace{1em}
Plik wyjściowy ma następująco strukturę:
\begin{enumerate}
    \item W kolejnych wierszach za wyjątkiem ostatniego wiersza zawarte są dane, dotyczące kolejnych połączeń producent-apteka, które zapewniają najmniejszy możliwy koszt całkowity. Każde połączenie (tranzakcja) między producentem a apteką kończy się znakiem nowej lini. Każde połączenie jest unikalne (w pliku nie wystąpi więcej niż jedno połączenie dotyczące tego samego producenta i tej samej apteki). Każda apteka może łączyć się z więcej niż jednym producentem, ale nie musi łączyć się z każdym. Nie wszyscy producenci z pliku wejściowego muszą pojawić się w pliku wyjściowym (tranzkacje z niektórymi producentami mogą być nieopłacalne).
    \item Ostatni wiersz pliku wyjściowego zawiera inforamację o wysokości całkowitego kosztu, jaki zapewniają (minimalizują) połączenia podane w wierszach poprzedzających. 
    \item Każdy wiersz będący połączeniem zawiera dane w następującej postaci, zachowując podaną kolejność: \emph{nazwa producenta -\textgreater\space nazwa apteki} [Koszt = \emph{ilość szczepionek kupionych od danego producenta*koszt pojedynczej szczepionki = koszt tranzakcji} zł]. \emph{Ilość szczepionek kupionych od danego producenta} jest liczbą całkowitą. \emph{Koszt pojedynczej szczepionki} oraz \emph{koszt tranzakcji} są liczbami zmiennoprzecinkowymi.
    \item Ostatni wiersz ma postać: Opłaty całkowite: \emph{koszt całkowity} zł
\end{enumerate}
% Nieprawdiłowe korzystanie z programu
\section{Nieprawidłowe użytkowanie programu}
Zarówno plik wejściowy może zawierać błędy, jak i samo wywołanie programu może być nieprawidłowe. W takich sytuacjach program będzie informował użytkownika o błędach odpowiednimi komunimatami. Scenariusze takich sytuacji przedstawiono poniżej:
\begin{enumerate}
    \item Brak wartości (danej):\\\\*
    \framebox{%
        \begin{minipage}{0.98\linewidth}
        Błąd w pliku wejśćiowym w lini 10: Brak wymaganej wartości.
        \end{minipage}}
        
    \vspace{1em}
    \item Nieodpowiedni typ danej:\\\\*
    \framebox{%
        \begin{minipage}{0.98\linewidth}
        Błąd w pliku wejśćiowym w lini 11: Nieodpowiedni typ, wymagana liczba całkowita. 
        \end{minipage}}
        
    \vspace{1em}
    \item Nieprawidłowa ilość znaków strumienia:\\\\*
    \framebox{%
        \begin{minipage}{0.98\linewidth}
        Błąd w pliku wejśćiowym w lini 12: Nieprawidłowa liczba znaków strumienia, wymgane 2 znaki strumienia. 
        \end{minipage}}
        
    \vspace{1em}
    \item Brak znaku '\#':\\*\\*
    \framebox{%
        \begin{minipage}{0.98\linewidth}
        Błąd w pliku wejściowym: Nieprawidłowa liczba sekcji lub brak znaku '\#'.
        \end{minipage}}
        
    \vspace{1em}
    \item Pwtórzenie w kolejnym wierszu tej samej apteki/producenta:\\*\\*
    \framebox{%
        \begin{minipage}{0.98\linewidth}
        Błąd w pliku wejściowym w lini 13: ID o danej wartości już wystąpiło. ID musi być wartością unikalną. 
        \end{minipage}}
        
    \vspace{1em}
    \item Pwtórzenie w kolejnym wierszu połączenia między tym samym producentem i tą samą apteką:\\*\\*
    \framebox{%
        \begin{minipage}{0.98\linewidth}
        Błąd w pliku wejściowym w lini 14: Połączenie tego producenta i tej apteki już wystąpiło. Połączenia muszą być unikalne. 
        \end{minipage}}
        
    \vspace{1em}
    \item Wprowadzenie wartości ujemnej w pliku:\\*\\*
    \framebox{%
        \begin{minipage}{0.98\linewidth}
        Błąd w pliku wejściowym w lini 15: Wartość ujemna, wymagana wartość dodatnia. 
        \end{minipage}}
        
    \vspace{1em}
    \item Brak podanego pliku podczas uruchomienia programu:\\*\\*
    \framebox{%
        \begin{minipage}{0.98\linewidth}
        Nieprawidłowa liczba argumentów wejściowych.
        \end{minipage}}
        
    \vspace{1em}
    \item Nieistniejący plik wejściowy:\\*\\*
    \framebox{%
        \begin{minipage}{0.98\linewidth}
        Nie można znaleźć pliku "data.txt" w bieżącym katalogu.
        \end{minipage}}

    \vspace{1em}
\end{enumerate}

\end{document}
