\documentclass{article}

% Formatting
\usepackage{polski}
\usepackage[utf8]{inputenc}
\usepackage[margin=1in]{geometry}

% Title content
\title{Algorytmy i struktury danych
\\Projekt indywidualny
\bigbreak Specyfikacja funkcjonalna}
\author{Kacper Baczyński, 276409}
\date{Warszawa, 2020}

\begin{document}

\maketitle

% Wymagania dla pliku wejściowego
\section{Wymagania dla pliku wejściowego}
Przykładowy plik wczytywany przez program powinien wyglądać następująco:\\*\\*
\begin{tabular}{l}
\# Producenci szczepionek (id \textbar\space nazwa \textbar\space dzienna produkcja)\\
0 \textbar\space BioTech 2.0 \textbar\space 900\\
1 \textbar\space Eko Polska 2020 \textbar\space 1300\\
2 \textbar\space Post-Covid Sp. z o.o. \textbar\space 1100\\
\# Apteki (id \textbar\space{} nazwa \textbar\space{} dzienne zapotrzebowanie)\\
0 \textbar\space CentMedEko Centrala \textbar\space 450\\
1 \textbar\space CentMedEko 24h \textbar\space 690\\
2 \textbar\space CentMedEko Nowogrodzka \textbar\space 1200\\
\# Połączenia producentów i aptek (id producenta \textbar\space id apteki \textbar\space dzienna maksymalna\\ liczba dostarczanych szczepionek \textbar\space koszt szczepionki [zł] )\\
0 \textbar\space 0 \textbar\space 800 \textbar\space 70.5\\
0 \textbar\space 1 \textbar\space 600 \textbar\space 70\\
0 \textbar\space 2 \textbar\space 750 \textbar\space 90.99\\
1 \textbar\space 0 \textbar\space 900 \textbar\space 100\\
1 \textbar\space 1 \textbar\space 600 \textbar\space 80\\
1 \textbar\space 2 \textbar\space 450 \textbar\space 70\\
2 \textbar\space 0 \textbar\space 900 \textbar\space 80\\
2 \textbar\space 1 \textbar\space 900 \textbar\space 90\\
2 \textbar\space 2 \textbar\space 300 \textbar\space 100\\*\\*
\end{tabular}\\
Wymagania dla pliku wczytywnaego przez program:
\begin{enumerate}
    \item Plik wejściowey musi posiadać trzy sekcje w następującej kolejności: \emph{producenci szczepionek, apteki, połączenia producentów i aptek}.
    \item Każda sekcja musi rozpoczynać się od znaku '\#'. Po tym znaku w tym samym wierszu sekcja może (ale nie musi) posiadać swoją nazwę oraz nazwy danych, które zawiera każdy następny wiersz tej sekcji. Po tytule sekcji ("\#...") konieczne jest przejście do nowej lini (znak nowej linii).
    \item Sekcja pierwsza - producenci szczepionek po wierszu tytułowym ("\#...") w kolejnych wierszach musi zawierać dane dotyczące kolejnych producentów. Dane producentów muszą być odzielone znakiem nowej lini (każdy producent to nowy wiersz). Dane każdego producenta muszą być podane w następującej kolejności: \emph{id \textbar\space nazwa \textbar\space dzienna produkcja}. W każdym wierszu po każdej danej musi wystąpić znak strumienia '\textbar' za wyjątkiem ostatniej danej w wierszu - \emph{dzienna produkcja}. Dana \emph{id} oznacza numer identyfikacyjny producenta i musi być liczbą całkowitą. Każdy producent musi mieć unikalne \emph{id} (dwóch producentów nie może mieć takiego samego \emph{id}). \emph{Nazwa} producenta może zawierać wszystkie znaki oprócz znaku strumienia '\textbar'. \emph{Dzienna produkcja} jaką osiąga dany producent musi być liczbą całkowitą.
    \item Sekcja druga - \emph{apteki} po wierszu tytułowym ("\#...") w kolejnych wierszach musi zawierać dane dotyczące kolejnych aptek. Dane aptek muszą być odzielone znakiem nowej lini (każda apteka to nowy wiersz). Dane każdej apteki muszą być podane w następującej kolejności: \emph{id \textbar\space nazwa \textbar\space dzienne zapotrzebowanie}. W każdym wierszu po każdej danej musi wystąpić znak strumienia '\textbar' za wyjątkiem ostatniej danej w wierszu - \emph{dzienne zapotrzebowanie}. Dana \emph{id} oznacza numer identyfikacyjny apteki i musi być liczbą całkowitą. Każda apteka musi mieć unikalne \emph{id} (dwie apteki nie mogą mieć takiego samego \emph{id}). \emph{Nazwa} apteki może zawierać wszystkie znaki oprócz znaku strumienia '\textbar'. \emph{Dzienne zapotrzebowanie} danej apteki musi być liczbą całkowitą.
    \item Sekcja trzecia - \emph{połączenia producentów i aptek} po wierszu tytułowym ("\#...") w kolejnych wierszach musi zawierać dane dotyczące kolejnych umów między danym producentem, a daną apteką. Dane połączeń muszą być odzielone znakiem nowej lini (każde połączenie to nowy wiersz). Sekcja ta musi zawierać połączenia (umowy) każdego producenta z każdą apteką. Dane każdej umowy muszą być podane w następującej kolejności: \emph{id producenta \textbar\space id apteki \textbar\space dzienna maksymalna liczba dostarczanych szczepionek \textbar\space koszt szczepionki [zł]}. W każdym wierszu po każdej danej musi wystąpić znak strumienia ‘\textbar’ za wyjątkiem ostatniej danej w wierszu - \emph{koszt szczepionki [zł]}.  Dane \emph{id producenta} oraz \emph{id apteki} to odpowiednio numer identyfikacyjny producenta i apteki z wcześniej omawianych sekcji, a więc dane te muszą spełniać wymagania wyszczególnione w tych sekcjach oraz mogą przyjmować tylko takie wartości jakie zostały w tych sekcjach wymienione (nie może być umowy między niezdefiniowaną wcześniej apteką czy prducentem). \emph{Dzienna maksymalna liczba dostarczanych szczepionek} wynikająca z umowy musi być liczbą całkowitą. \emph{Koszt szczepionki [zł]} po jakiej dany producent sprzedaje szczepionkę danej aptece może być liczbą zmiennoprzecinkową. Koszt ten musi być podany w złotych [zł].
\end{enumerate}
% Instrukcja korzystania z programu
\section{Instrukcja korzystania z programu}
% Wynik działania programu (plik wyjściowy)
\section{Wynik działania programu (plik wyjściowy)}
\end{document}
